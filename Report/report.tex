\documentclass{article}
\usepackage{hyperref}
\usepackage[procnames]{listings}
\usepackage{color}
\title{Assignment 4}
\date{2016-02-25}
\author{Mohammed Shaaban}
\usepackage[pdftex]{graphicx}
\usepackage{listings}
\usepackage{alltt}


\definecolor{dkgreen}{rgb}{0,0.6,0}
\definecolor{gray}{rgb}{0.5,0.5,0.5}
\definecolor{mauve}{rgb}{0.58,0,0.82}

\lstset{
	basicstyle=\footnotesize,
	breaklines=true,
}

\begin{document}
  \maketitle
\section{Problem 1}\label{di:di}
In the first question I downloaded the XML file from the website using python that contains four defs(functions). First function contains the piece of code that gets the line from the XML file having the word ``friends count'' but as a whole string. Then I create another function to get the number that is specified withthe `` friend count'' and put it in a text file. Finally I perform the mean,standard deviation and median calculations to get the requirements.

The following is the piece of code that does all of that:


\lstinputlisting[language=Python,frame=single,caption={python code to get mean and standard deviation},label=lst:q2code1,captionpos=b,numbers=left,
numberstyle=\tiny\color{gray},
  keywordstyle=\color{blue},
  commentstyle=\color{dkgreen},
  stringstyle=\color{mauve},showspaces=false,showstringspaces=false,basicstyle=\footnotesize]{Q1-A4.py}



\begin{figure}
\centering
\includegraphics[scale=0.85]{an.png}
\caption{Standard deviation}
\label{fig:figq.png}
\end{figure}
\newpage
From the file ``numbers count.txt'' I drawn the graph above:
\begin{figure}
\centering
\includegraphics[scale=0.25]{ap.png}
\caption{Grapgh in R}
\label{fig:fiwg.png}
\end{figure}

\section{Problem 2}

In problem 2, using Twitter API we extract the followers from twitter account after that I use the same program to develop the standard deviation and median to get the calculations based on every and each follower I have in the list.

The python code to get the followers is as follow:
\lstinputlisting[language=Python,frame=single,caption={python code to get mean and standard deviation},label=lst:q2code1,captionpos=b,numbers=left,
numberstyle=\tiny\color{gray},
  keywordstyle=\color{blue},
  commentstyle=\color{dkgreen},
  stringstyle=\color{mauve},showspaces=false,showstringspaces=false,basicstyle=\footnotesize]{Q2-1a4.py}


\begin{figure}
\centering
\includegraphics[scale=0.35]{mo.png}
\caption{running the same program created a text file as shown in section  \ref{di:di}. we perform on this text file exactly what we did on the text file to get these numbers}
\label{fig:fig.png}
\end{figure}



\begin{figure}
\centering
\includegraphics[scale=0.35]{a7a.png}
\caption{Finally the graph represented in the following figure based on the calculations}
\label{fig:a7a.png}
\end{figure}

\end{document}